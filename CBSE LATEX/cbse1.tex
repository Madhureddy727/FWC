\documentclass{article}
\usepackage{graphicx}
\usepackage{amsmath}
\usepackage{gvv}
\usepackage{float}
\usepackage{enumitem}
\begin{document}
\begin{enumerate}
\section{CBSE}
\subsection{geometry}
\item  In what ratio, does x-axis divide the line segment joining the points $A (3,6)$ and $B (-12,-3)$ ?  
\begin{enumerate} 
\item $1 : 2$
\item $1 : 4$
\item $4 : 1$ 
\item $2 : 1$
\end{enumerate}
\item In the given figure\figref{fig:fig2} $PQ$ is tangent to the circle centered at $O$. If $\angle AOB = 95^\circ$, then the measure of $\angle ABQ$ will be
\begin{figure}[H] 
\centering
\includegraphics[width=\columnwidth]{./images_latex/madhuimage2.jpg}  
\label{fig:fig2}
\caption{circle with triangle}
\end{figure}
\item  Curved surface area of a cylinder of height $5cm$ is $94.2cm^2$. Radius of the cylinder is (Take$\pi=3.14$) 
\begin{enumerate} 
\item $2 cm$
\item $3 cm$
\item $2.9 cm$ 
\item $6 cm$
\end{enumerate}
 \item The curved surface area of a cone having height $24 cm$ and radius $7 cm$, is
 \begin{enumerate}
 \item $528 cm^2$ 
 \item $1056 cm^2$
 \item $550 cm^2$ 
 \item $500 cm^2$ 
 \end{enumerate}
 \item  The distance between the points $(0,2\sqrt{5})$ and $(-2\sqrt{5},0)$ is 
 \begin{enumerate} 
 \item $2\sqrt{10}$ units 
 \item $4\sqrt{10}$ units 
 \item $2\sqrt{20}$ units 
 \item $0$
 \end{enumerate}
 \item  \textbf{Assertion (A)}: Point P $(0,2)$ is the point of intersection of y-axis with the line ${3x+2y=4}$ \\
 \textbf{ Reason (R)}: The distance of point p $(0,2)$ from x-axis is $2$ units.
\item \textbf{Assertion (A)}: The perimeter of  $\Delta {ABC}$  is a rational number.\\
\textbf{Reason (R)}: The sum of the squares of two rational numbers is always rational. 
\begin{figure}[H] 
\centering 
\includegraphics[width=\columnwidth]{./images_latex/madhuimage3.jpg}  
\label{fig:fig3} 
\caption{Triangle ABC}
\end{figure}
\item In the given figure\figref{fig:fig4} $XZ$ is parallel to $BC$. $AZ=3 cm$, $ZC=2 cm$, $BM=3 cm$, and $MC=5 cm$. Find the length of $XY$.
\begin{figure}[H] 
\centering
\includegraphics[width=\columnwidth]{./images_latex/madhuimage4.jpg}  
\label{fig:fig4} 
\caption{ Multiple Triangels}
\end{figure}
\item If $(-5, 3)$ and $(5, 3)$ are two vertices of an equilateral triangle, then find coordinates of the third vertex, given that the origin lies inside the triangle. (Take $\sqrt{3} = 1.7$)
\item Two tangents $TP$ and $TQ$ are drawn to a circle with center $O$ from an external point $T$. Prove that $\angle PTQ = 2\angle OPQ$ 
\begin{figure}[H] 
\centering
\includegraphics[width=\columnwidth]{./images_latex/madhuimage5_1.jpg}  
\label{fig:fig5}
\caption{circle open triangle PQT}
\end{figure}
\item In the given figure\figref{fig:fig6} a circle is inscribed in a quadrilateral $ABCD$ in which $\angle B=90^\circ$. If $AD=17 cm$, $AB=20 cm$, and $DS=3 cm$, then find the radius of the circle.
\begin{figure}[H]
\centering
\includegraphics[width=\columnwidth]{./images_latex/madhuimage6.jpg}  
\label{fig:fig6}
\caption{Quadrilateral circle}
\end{figure}
\item A room is in the form of a cylinder surmounted by a hemi-spherical dome. The base radius of the hemisphere is one-half the height of the cylindrical part. Find the total height of the room if it contains  $(\frac{1408}{21})$ $m^3$ of air.(Take$\pi = \frac{22}{7}$)
\item An empty cone is of radius $3 cm$ and height $12 cm$. Ice-cream is filled in it so that the lower part of the cone, which is $(\frac{1}{6})^th$ of the volume of the cone, is unfilled, but a hemisphere is formed on the top. Find the volume of the ice-cream. (Take $\pi = 3.14$) 
\begin{figure}[H]
\centering
\includegraphics[width=\columnwidth]{./images_latex/madhuimage7.jpg}
\label{fig:fig7}
\caption{Cone}
\end{figure}
\item If a line is drawn parallel to one side of a triangle to intersect the other two sides at distinct points, prove that the other two sides are divided in the same ratio.
\item  The angle of elevation of the top of a tower $24m$ high from the foot of another tower in the same plane is $60^\circ$. The angle of elevation of the top of the second tower from the foot of the first tower is $30^\circ$. Find the distance between two towers and the height of the other tower. Also, find the length of the wire attached to the tops of both the towers.
\item A spherical balloon of radius r subtends an angle of $60^\circ$ at the eye of an observer. If the angle of elevation of its centre is $45^\circ$ from the same point, then prove that the height of the centre of the balloon is $\sqrt{2}$ times its radius. 
\item  A chord of a circle of radius $14 cm$ subtends an angle of $60^\circ$ at the centre. Find the area of the corresponding minor segment of the circle. Also, find the area of the major segment of the circle.
\item The discus throw is an event in which an athlete attempts to throw a discus. The athlete spins anti-clockwise around one and a half times through a circle,then releases the throw. When released, the discus travels along a tangent to the circular spin orbit                                                         
\begin{figure}[H]
\centering  
\includegraphics[width=\columnwidth]{./images_latex/madhuimage11.jpg}
\label{fig:fig11}
\caption{Discus Throw}
\end{figure}
In the given figure\figref{fig:fig12} $AB$  is one such tangent to a circle of radius $75 cm$. Point O  is the centre of the circle and  $\angle ABO = 30^\circ$ . $PQ$  is parallel to  $OA$ .
\begin{figure}[H]
\centering
\includegraphics[width=\columnwidth]{./images_latex/madhuimage12.jpg}
\label{fig:fig12}
\caption{circle with radius}
\end{figure}
Based on the above information:
\begin{enumerate}
\item Find the length of AB.
\item Find the length of OB.
\item Find the length of AP.
\item Find the length of PQ. 
\end{enumerate}
\subsection{data handling}
\item  The distribution below gives the marks obtained by 80 students on a test:
\begin{table}[htb]
\centering
\resizebox{\columnwidth}{!}{
\begin{tabular}{|c|c|c|c|c|c|c|c|}
\hline
\textbf{Marks} & Less than 10 & Less than 20 & Less than 30 & Less than 40 & Less than 50 & Less than 60 \\ \hline
\textbf{Number of Students} & 3 & 12 & 27 & 57 & 75 & 80 \\ \hline
\end{tabular}
		}
\end{table}\\
The model class of this distribution is:
\begin{enumerate}
\item $10 - 20$
\item $20 - 30$ 
\item $30 - 40$
\item $50 - 60$
\end{enumerate}
\item If the value of each observation of a statistical data is increased by $3$, then the mean of the data  
\begin{enumerate}
\item remains unchanged  
\item increases by $3$  
\item increases by $6$
\item increases by $3n$
\end{enumerate}
\item India Meteorological Department observes seasonal and annual rainfall every year in different sub-divisions of our country.
\begin{figure}[H]  
\centering
\includegraphics[width=\columnwidth]{./images_latex/madhuimage10.jpg}
\label{fig:fig10}
\caption{Meteorological departmaent logo} 
\end{figure}
It helps them to compare and analyse the results. The table given below shows sub-division wise seasonal (monsoon) rainfall (mm) in $2018$:
\begin{center}
\begin{tabular}{|c|c|}
\hline
\textbf{Rainfall (mm)} & \textbf{Number of Sub-divisions} \\ \hline
 200-400 & 2 \\ \hline
 400-600 & 4 \\ \hline
 600-800 & 7 \\ \hline
 800-1000 & 4 \\ \hline
 1000-1200 & 2 \\ \hline
 1200-1400 & 3 \\ \hline
 1400-1600 & 1 \\ \hline
 1600-1800 & 1 \\ \hline
 \end{tabular}
 \end{center}
\begin{enumerate}
\item Write the modal class.
\item Find the median of the given data.
Find the mean rainfall in this season.
\item If a sub-division having at least $1000mm$ rainfall during monsoon season, is considered a good rainfall sub-division, then how many sub-divisions had good rainfall?
\end{enumerate}
\subsection{algebra}
\item The graph of $y=p(x)$ is given, for a polynomial $p(x)$.The number of zeroes of $p(x)$ from  the graph is 
\begin{figure}[H]  
\centering
\includegraphics[width=\columnwidth]{./images_latex/madhuimage1.jpg} 
\label{fig:fig1}
\caption{Graph }
\end{figure}
\begin{enumerate}[label=(\alph*)]
\item $3$
\item $1$
\item $2$
\item $0$ 
\end{enumerate}
\item The value of $k$ for which the pair of equations $kx=y+2$ and $6x=2y+3$ has infinitely many solutions
\begin{enumerate}
\item is $k=3$
\item does not exist
\item is $k=-3$
\item is $k=4$
\end{enumerate}
\item If $\alpha, \beta$ are the zeroes of a polynomial $p(x) = x^2 + x-1$, then $\frac{1}{\alpha} + \frac{1}{\beta}$ equals to
\begin{enumerate}
\item $1$ 
\item $2$
\item $-1$ 
\item $\frac{-1}{2}$
\end{enumerate}
\item The least positive value of $k$, for which the quadratic equation $2x^2+kx-4=0$ has rational  roots, is 
\begin{enumerate} 
\item $\pm 2\sqrt{2}$ 
\item $2$
\item $\pm 2$
\item $\sqrt{2}$
\end{enumerate}
\item Which of the following is a quadratic polynomial having zeroes $\frac{-2}{3}$ and $\frac{2}{3}$ ?
\begin{enumerate}  
\item $4x^2 - 9$
\item $\frac{4}{9}(9x^2 + 4)$
\item $x^2 + \frac{9}{4}$
\item $5(9x^2 - 4)$ 
\end{enumerate}
\item Solve the pair of equations $x = 3$ and $y = -4$ graphically.
\item Using graphical method, find whether the following system of linear equations is consistent or not:\\
	$x = 0$ and $y = -7$
\item Find the greatest number which divides $85$ and $72$ leaving remainders $1$ and $2$ respectively.
\item Half of the difference between two numbers is $2$. The sum of the greater number and twice the smaller number is $13$. Find the numbers.
\item Prove that $\sqrt{5}$ is an irrational number.
\item While designing the school yearbook, a teacher asked the student that the length and width of a particular photo is increased by $x$ units each to double the area of the photo. The original photo is $18 cm$ long and $12 cm$ wide. 
\begin{enumerate} 
\item Write an algebraic equation depicting the above information.
\item Write the corresponding quadratic equation in standard form. 
\item What should be the new dimensions of the enlarged photo? 
\begin{figure}[H]
\centering 
\includegraphics[width=\columnwidth]{./images_latex/madhuimage9.jpg}   
\label{fig:fig9}
\caption{school}
\end{figure} 
\end{enumerate}
Can any rational value of $x$ make the new area equal to $220 cm^2$?
\subsection{probability}
\item  Probability of happening of an event is denoted by $p$ and probability of non-happening of the event is denoted by $q$. The relation between $p$ and $q$ is 
\begin{enumerate}
\item $p + q = 1$
\item $p = 1,q = 1$ 
\item $p = q -1$ 
\item $p +q +1 = 0$ 
\end{enumerate} 
\item A girl calculates that the probability of her winning the first prize in a lottery is $0.08$. If $6000$ tickets are sold, how many tickets has she bought ? 
\begin{enumerate}   
\item $40$ 
\item $240$
\item $480$
\item $750$
\end{enumerate}
\item In a group of $20$ people, $5$ can't swim. If one person is selected at random, then the probability that he/she can swim, is
\begin{enumerate}
\item $\frac{3}{4}$
\item $\frac{1}{3}$
\item $1$
\item $\frac{1}{4}$
\end{enumerate}
\item A bag contains $4$ red, $3$ blue, and $2$ yellow balls. One ball is drawn at random from the bag. Find  the probability that the drawn ball is  
\begin{enumerate}
\item red
\item yellow.
\end{enumerate}
\subsection{sequences}
\item If $p-1$, $p+1$ and $2p+3$ are in A.P., then the value of $p$ is 
\begin{enumerate} 
\item $-2$
\item $4$
\item $0$
\item $2$ 
\end{enumerate}
\item The ratio of the $11^th$ term to the $17^th$ term of an A.P. is $3:4$. Find the ratio of the $5^th$ term to the $21^st$ term of the same A.P. Also, find the ratio of the sum of the first $5$ terms to that of the first $21$ terms.
\item $250$ logs are stacked in the following manner: $22$ logs in the bottom row, $21$ in the next row, $20$ in the row next to it, and so on (as shown by an example). In how many rows are the $250$ logs placed and how many logs are there in the top row ?  
\begin{figure}[H] 
\centering
\includegraphics[width=\columnwidth]{./images_latex/madhuimage8.jpg}
\label{fig:fig8}
\caption{logs}
\end{figure}
\subsection{trignometry}
\item If $2\tan A=3$, then the value of  
\begin{align}
\frac{4\sin A+3\cos A} {4 \sin A-3\cos A} \\
\end{align} 
\begin{enumerate}
\item $\frac{7}{\sqrt{13}}$ 
\item $\frac{1}{\sqrt{13}}$
\item $3$ 
\item does not exist
\end{enumerate}
\item $\left[\frac{3}{4} \tan^2 30^\circ - \sec^2 45^\circ + \sin^2 60^\circ\right]$ is equal to
\begin{enumerate}  
\item $-1$ 
\item $\frac{5}{6}$ 
\item $\frac{-3}{2}$  
\item $\frac{1}{6}$ 
\end{enumerate}
\item If $\sin \theta + \cos \theta=\sqrt{3}$, then find the value of $\sin \theta . \cos \theta$. 
\item If $\sin \alpha = \frac{1}{\sqrt{2}}$ and $\cot \beta = \sqrt{3}$, then find the value of $\csc \alpha + \csc \beta$.
\item Prove that:  
\begin{align}
\frac{\tan \theta + \sec \theta - 1} {\tan \theta - \sec \theta + 1} = \frac{1 +\sin \theta} {\cos \theta} 
\end{align}
\end{enumerate}	
\end{document}
